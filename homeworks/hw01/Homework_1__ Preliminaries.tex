\documentclass{article}
\usepackage[margin=0.75in]{geometry}
\usepackage{graphicx}
\usepackage{enumitem}
\usepackage{natbib}
\usepackage{multicol}
\usepackage{hyperref}
\hypersetup{
	colorlinks=true,
	linktoc=all,
	linkcolor=blue,
}

\RequirePackage{amsmath, amssymb, amsthm}
\theoremstyle{definition}
\newtheorem{problem}{Problem}

\usepackage{listings}
\lstdefinestyle{mystyle}{
    backgroundcolor=\color{white},   
    commentstyle=\color{codegreen},
    keywordstyle=\color{magenta},
    numberstyle=\tiny\color{codegray},
    stringstyle=\color{codepurple},
    basicstyle=\ttfamily\footnotesize,
    breakatwhitespace=false,         
    breaklines=true,                 
    captionpos=b,                    
    keepspaces=true,                 
    numbers=none,                    
    numbersep=5pt,                  
    showspaces=false,                
    showstringspaces=false,
    showtabs=false,                  
    tabsize=2
}

\lstset{style=mystyle}

\title{EE 656 - Homework 1 - Preliminaries }

 \author{}

\begin{document}
\maketitle


This problem set counts for about $7 \%$ of your course grade. You are encouraged to talk at the conceptual level with other students, but you must complete all work individually and may not share any non-trivial code or solution steps. See the website for the full collaboration policy.

\section*{Submission Instructions}
Your assignment must be received by $11: 59 \mathrm{pm}$ on Friday, January 23. You are to upload your assignment directly to the Gradescope website as two attachments:


\begin{enumerate}
  \item A .tar.gz or .zip file containing a directory named after your ID with the structure shown below.


\begin{verbatim}
ID_hw1.tgz:
ID_hwl/
ID_hw1/task2c.m
\end{verbatim}

Or a Jupyter notebook per task using Python or Julia kernels for your programming. Follow the same naming convention task2c\_python.ipynb or task2c\_julia.ipynb for the notebooks.

\item A PDF with the written portion of your write-up. Scanned versions of hand-written documents, converted to PDFs, are perfectly acceptable. No other formats (e.g., .doc) are acceptable. Your PDF file should adhere to the following naming convention: ID\_hw1.pdf.

\end{enumerate}
\section*{1 Probability Basics (30 points)}
\begin{enumerate}[label=\Alph*.]
\item (5 pts) State the definition of $\mathbb{E}[X+Y+Z]$ in terms of the underlying probability distribution, $p(x, y, z)$, where the random variables are continuous. Then prove that $\mathbb{E}[X+Y+Z]=\mathbb{E}[X]+\mathbb{E}[Y]+\mathbb{E}[Z]$, using the definition of expectation. Your proof should be succinctly and clearly written. Do NOT assume that the random variables are independent.\\
\item ( 5 pts ) For each of the covariance matrices below, indicate whether it is Valid or Invalid. If it is invalid, give a reason.\\
\begin{enumerate}[label=(\alph*)]
    \item $\left[\begin{array}{cc}8 & -5 \\ -5 & 10\end{array}\right]$
    \item $\left[\begin{array}{cc}3 & 7 \\ 7 & -5\end{array}\right]$
    \item $\left[\begin{array}{ll}8 & 5 \\ 3 & 3\end{array}\right]$
    \item $\left[\begin{array}{ll}3 & 7 \\ 7 & 4\end{array}\right]$
    \item $\left[\begin{array}{ll}5 & 4 \\ 4 & 8\end{array}\right]$
\end{enumerate}
\item (12 pts) Ahmed has collected four points (in 2D space); let's call this set '$r$'. He has computed their mean and (biased) sample covariance to be:

$$
\mu_{r}=\left[\begin{array}{l}
2 \\
3
\end{array}\right] \quad \Sigma_{r r}=\left[\begin{array}{ll}
5 & 3 \\
3 & 7
\end{array}\right]
$$

And Fahd has collected six points (from the same 2D space); let's call it set '$m$', finding their mean and (biased) sample covariance to be:

$$
\mu_{m}=\left[\begin{array}{c}
-2 \\
2
\end{array}\right] \quad \Sigma_{m m}=\left[\begin{array}{ll}
8 & 4 \\
4 & 3
\end{array}\right]
$$

\begin{enumerate}[label=(\alph*)]
\item If Ahmed computed the sum (over all $r$ points) of $x x^{\top}$, what value would he have computed? (Show work and provide a numerical answer.)
\item If Fahd computed the sum (over all $m$ points) of $x x^{\top}$, what value would he have computed? (Show work and provide a numerical answer.)
\end{enumerate}
We now wish to compute the mean and sample covariance of all ten points (which we will denote $m+r$ ).
\begin{enumerate}
    \setcounter{enumii}{2}
    \item What is the mean, $\mu_{m+r}$ ? (Show work and provide a numerical answer.)
    \item What is the (biased) sample covariance, $\Sigma_{m+r}$ ? (Show work and provide a numerical answer.)
\end{enumerate}

Note: Biased sample covariance has $\frac{1}{N}$ whereas the unbiased version has $\frac{1}{N-1}$. The question is just making it easier by using the biased formula. See Wikipedia page of Bessel's correction for more explanation.\\

\item (8 pts) Answer the following. Any proofs should be succinct and clear.\\
\begin{enumerate}
    \item Prove that if A and B are independent events, so are $A^{C}$ and $B^{C} . A^{C}$ is the complement of A such that $\mathrm{A} \cap A^{C}=\varnothing$.
    \item Prove that the affine transformation of a Gaussian random vector is Gaussian.
\end{enumerate}
\end{enumerate}


\section*{2 Bayes Filter (30 points)}
\begin{enumerate}[label=\Alph*.]
\item (10 pts) Suppose that $1 \%$ of the general population has cancer and that a particular test for cancer has a $20 \%$ false positive rate and a $10 \%$ false negative rate. Your test result is positive. What is the probability that you indeed have cancer?

Suppose now that the National Institute of Health has $\$ 1,000,000$ of research funding to invest in improving this test. In this particular instance which would have increase test accuracy the most and why: reducing the false positive rate or reducing the false negative rate?

\item (10 pts) A robot is equipped with a manipulator to paint an object. Furthermore, the robot has a sensor to detect whether the object is colored or blank. Neither the manipulation unit nor the sensor are perfect.

From previous experience you know that the robot succeeds in painting a blank object with a probability of

$$
p\left(x_{t+1}=\text { colored } \mid x_{t}=\text { blank, } u_{t+1}=\text { paint }\right)=0.9
$$

where $x_{t+1}$ is the state of the object after executing a painting action, $u_{t+1}$ is the control command, and $x_{t}$ is the state of the object before performing the action.

The probability that the sensor indicates that the object is colored although it is blank is given by $p(z=$ colored $\mid x=$ blank $)=0.2$, and the probability that the sensor correctly detects a colored object is given by $p(z=$ colored $\mid x=$ colored $)=0.7$.

Unfortunately, you have no knowledge about the current state of the object. However, after the robot performed a painting action the sensor of the robot indicates that the object is colored.

Compute the probability that the object is blank after the robot has performed an action to paint it. Use an appropriate prior distribution and justify your choice.

\item (10 pts) Consider a robot that resides in a circular world consisting of ten different places that are numbered counterclockwise. The robot is unable to sense the number of its present place directly. However, places 0, 3, and 6 contain a distinct landmark, whereas all other places do not. All three of these landmarks look alike. The likelihood that the robot observes the landmark given it is in one of these places is 0.8 . For all other places, the likelihood of observing the landmark is 0.4.

For each place on the circle, we wish to compute the probability that the robot is in that place given that the following sequence of actions is carried out deterministically and the following sequence of observations is obtained: The robot detects a landmark, moves 3 grid cells counterclockwise and detects a landmark, and then moves 4 grid cells counterclockwise and finally perceives no landmark.

Implement the circular world described above using a discrete Bayes filter in Matlab/Python/Julia to numerically arrive at the desired belief and report your resulting belief in the PDF. The coding template for Python (task2c\_python.ipynb) and Julia (task2c\_julia.ipynb) are provided. Follow the naming convention when submitting your code.

\end{enumerate}

\section*{3 Bayes Filter (12 points)}
Based on last year's test scores the following is the probability of certain scores.

$$
A=30 \%,  B=60 \%, C=10 \% .
$$

At this point in the exam, you typically have a good idea of how it's going. Your internal barometer tells you that it is going great. Based on your experience, you know the probabilities of you having a "great" feeling given different scores are as follow:

$$
A=70 \%, B=25 \%, C=5 \% .
$$

With the prior distribution and internal barometer measurement, calculate the posterior probability of getting an $\mathrm{A}, \mathrm{B}$ or C on the exam.

\section*{4 Normal Random Variables (8 points)}
If $X$ and $Y$ are jointly normal random variables with means $\mu_{x}$ and $\mu_{y}$ and variances $\sigma_{x}$ and $\sigma_{y}$, which of the following statements is/are necessarily true.\\
Answer \textbf{T / F / $<$blank$>$} to each question below. Correct answers earn two point; blanks earn zero; incorrect answers earn minus two. The total score is $\max (0$, sum of parts $a-d)$.
\begin{enumerate}[label=(\alph*)]
    \item $ \sigma_{X \mid Y}=\sigma_{X}$.
    \item $\sigma_{X \mid Y}<=\sigma_{X}$.
    \item $\sigma_{X \mid Y}<=\sigma_{X}$ or $\sigma_{X \mid Y}>=\sigma_{X}$ depending on the correlation between X and Y .
    \item $\sigma_{X \mid Y}<=\sigma_{X}$ and the value of $\sigma_{X \mid Y}$ does not depend on the cross-covariance between $X$ and $Y$.
\end{enumerate}

\section*{5 Cholesky Decomposition (8 points)}
For each of the matrices below, does the matrix have a unique Cholesky decomposition? If not, why? If so, compute the lower triangular matrix $L$ from the decomposition by hand (show work). If not, state why. Hint: $M=L L^{\top}$.
\begin{enumerate}[label=(\alph*)]
\item
$
\left[\begin{array}{ll}
6 & 5 \\
5 & 4
\end{array}\right]
$

\item$\left[\begin{array}{cc}25 & -10 \\ -10 & 40\end{array}\right] \quad$ 
\item$\left[\begin{array}{cc}3 & -6 \\ -4 & 6\end{array}\right] \quad$ 
\item$\left[\begin{array}{cc}9 & -3 \\ -3 & 5\end{array}\right] \quad$ 
\end{enumerate}
\section*{6 Probability and Uncertainty Propagation (12 points)}
A range finder sensor provides noisy range measurements $z \sim \mathcal{N}\left(\mu_{z}, \sigma_{z}^{2}\right)$. A corrected model for calibrating the sensor is $y=a z+b$, for some constant $a, b \in \mathbb{R}$. Derive the distribution over $y$. Is the noisy model for $y$ an approximation or exact? Why?


\end{document}